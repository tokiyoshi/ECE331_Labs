\footnotesize{
\textbf{With the hot probe apparatus, the voltmeter will read a positive voltage for $V_h\,-\,V_{c}$ if the material is n-type, and a negative voltage for $V_h\,-\,V_{c}$ if the material is p-type. Why?}
}
\normalsize
\\ \\ 
The hot($h$) point is thermally excited, due to this the majority carriers move away from from it to the cold($c$) point. This movement of the majority carriers will lead to a current \& hence a voltage difference between $V_c$ and $V_h$. The majority carrier will be electrons in the n-type material, which means the voltage difference will be positive, and in the p-type material the majority carrier will be holes which leads to a negative voltage difference.