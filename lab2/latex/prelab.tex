\renewcommand{\labelenumi}{\textbf{\roman{enumi})}}
\begin{enumerate}
    \item Defining the following:
        \begin{labeling}{The Fill Factor, font=\normalfont\itshape} \itemsep10pt
        \item [The Fill Factor] The ratio of maximum obtainable power to the product of the open-circuit voltage and short-circuit current. (FF $=\frac{V_{max} \cdot I_{max}}{V_{oc} \cdot I_{sc}}$) This is geometrically defined by the largest rectangle which will fit inside of the illuminated I-V curve and is a measure of efficiency where the higher the fill factor, the better the solar cell.
        
        \item [V$_{oc}$ and I$_{sc}$ ] V$_{oc}$ is the open circuit voltage which is also known as the no load voltage (I = 0), this is an equilibrium point where the forward current compensates for the reverse photo-current.
        %maximum voltage available from a solar cell 
        and I$_{sc}$ is the short circuited (V = 0) current, where the current is solely due to the collection of the optically generated carriers.
        %current through the solar cell when the voltage across the solar cell is zero
        \item [Solar cell efficiency] The solar cell efficiency refers to the portion of energy in the form of sunlight that can be converted via photovoltaics into electricity.
        \end{labeling}
    \item I-V characteristics curve of a typical solar cell:
    
    The I-V curve of a typical solar cell with no illumination is described by that of the usual diode graph, therefore it is an exponential graph with Voltage as the x-axis. When the light is applied to the cell, the exponential shape of the graph does not change but a negative baseline current is generated. Hence the graph crosses the y-axis at $I_{sc}$ due to the generated photoelectric current, and intersects the x-axis at V$_{oc}$.
    
\end{enumerate}


